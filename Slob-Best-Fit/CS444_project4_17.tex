\documentclass[draftclsnofoot, onecolumn, 10pt, compsoc]{IEEEtran}

\usepackage[english]{babel}
\usepackage{amsmath}
\usepackage{graphicx}
\usepackage[top=0.75in, bottom=0.75in, left=0.75in, right=0.75in]{geometry}

\usepackage{listings}
\usepackage{color}

\usepackage{hyperref}
\usepackage{enumitem}

\usepackage{tabu}
\usepackage{longtable}

\title{\textbf{Operating Systems II}\\Fall 2017\\Homework 3}

\author{Omeed Habibelahian\\Jeremy Fischer\\Group 17}

\begin{document}
	\maketitle
	\newpage
	
	\section{Design}
		\begin{lstlisting}
		scan through all pages
		  scan through all blocks
		    record the difference between size needed and free units available
			  If the difference is zero
			    use this spot
			    end
			 if the difference is smaller than the previous difference
			   use this location instead
		If there's no possible locations
		  create a new page
		\end{lstlisting}
		
	
	\section{Version Control Log}

	\section{Work Log}
		Our Work Log is the same as our Version Control Log above.
		One can see what was done when, as well as that we followed our design pretty tightly by looking at the commit messages.
		
	\section{Questions}
		\subsection{Main Point of the Assignment}

		\subsection{Our Approach}

		\subsection{Ensuring Correctness}

		\subsection{What We Learned}
			We learned how to\dots 
			\begin{itemize}
				\item use the Linux crypto API
				
				\item create a module
				
				\item load and unload a module
				
				\item create a block device
				
				\item mount a drive
			\end{itemize}
		\subsection{How to Evaluate and Prove Correctness}
			\textbf{Steps to Run with Patch}
			\begin{enumerate}
				\item git clone "git://git.yoctoproject.org/linux-yocto-3.19" linux-yocto-3.19-patched
				
				\item cd linux-yocto-3.19-patched
				
				\item git checkout v3.19.2
				
				\item Apply Patch
				\begin{enumerate}
					\item cp ../kernAssn3.patch drivers/block
				
					\item cd drivers/block
				
					\item patch $<$ kernAssn3.patch
				\end{enumerate}
				\item cd ../../
				
				\item cp /scratch/files/core-image-lsb-sdk-qemux86.ext4 .
				
				\item cp /scratch/files/config-3.19.2-yocto-standard .config
				
				\item make menuconfig
				
				\item Save and Exit
				
				\item make -j4 all
				
				\item cp /scratch/files/core-image-lsb-sdk-qemux86.ext4 ../
				
				\item Use the \textit{screen} command to create two split-screens.
				
				\item In both screens, make sure to source the environment variable located at /scratch/files.
				\begin{enumerate}
					\item{\textbf{Screen 1:}qemu-system-i386 -gdb tcp::5517 -S -nographic -kernel arch/x86/boot/bzImage -drive file=../core-image-lsb-sdk-qemux86.ext4,if=ide -enable-kvm -usb -localtime --no-reboot --append "root=/dev/hda rw console=ttyS0 debug"}
					
					\item \textbf{Screen 2:} \$GDB
					
					\item \textbf{Screen 2:} target remote :5517
					
					\item \textbf{Screen 2:} continue
					
					\item \textbf{Screen 1:} log in as root.
					
					\item The rest of the commands below reside in \textbf{Screen 1:}
					
					\item{scp \textbf{username}@os2.engr.oregonstate.edu:\textbf{your dir}/linux-yocto-3.19-patched/drivers/block/block\_dev\_crypto.ko /home/root}
					
					\item{Initialize the module and set the key and kylen: \textit{insmod block\_dev\_crypto.ko key="abcdefghijklmnop" keylen=16}}
					
					\item \textit{dmesg} In the output you should see that the module was initialized and the key was set
					
					\item \textit{dmesg -c} Clear the kernel 's output clutter
					
					\item Partition the disk:\textit{ fdisk /dev/crptblkd0}
						Follow these instructions when prompted
						\begin{itemize}
							\item Command (m for help): \textbf{n}
					
							\item Command action  e extended  p primary partition (1-4): \textbf{p}
					
							\item Partition number (1-4): \textbf{1}
					
							\item First sector(1-1023, default 1): \textbf{1}
					
							\item Last sector, +sectors or +size{K,M,G,T,P} (1-1023, default 1023): \textbf{$<$press enter$>$}
					
							\item Command (m for help): \textbf{w}
						\end{itemize}
					
					\item Create filesystem: \textit{mkfs.ext2 /dev/crptblkd0p1}
					
					\item Mount filesystem: \textit{mount /dev/crptblkd0p1 /mnt}
				\end{enumerate}
				
				\item Now that the module is loaded and the drive is mounted, we will assert its correctness
				\begin{itemize}
					\item \textit{echo AA $>$ /mnt/file1}
				
					\item \textit{ls -l /mnt} You should see file1 now
				
					\item 
						\textit{dmesg} It may take a few seconds, but you will see that "Before ENcryption" the data reads 41410000...0000....000. 
						The output is in hex, and 4141 is AA.
						in the "After ENcryption" section you can see that the data was encrypted to giberish now
					
					\item \textit{cat /mnt/file1} You will see AA as the output
				\end{itemize}
				
				\item Unmount the drive: \textit{umount /mnt}
				
				\item Uninstall the module: \textit{rmmod block\_dev\_crypto}
				
				\item reboot (this stops qemu)
			\end{enumerate}

\end{document}