\documentclass[draftclsnofoot, onecolumn, 10pt, compsoc]{IEEEtran}

\usepackage[english]{babel}
\usepackage{amsmath}
\usepackage{graphicx}
\usepackage[top=0.75in, bottom=0.75in, left=0.75in, right=0.75in]{geometry}

\usepackage{listings}
\usepackage{color}

\usepackage{hyperref}
\usepackage{enumitem}

\usepackage{tabu}
\usepackage{longtable}

\title{\textbf{Operating Systems II}\\Fall 2017\\Homework 4}

\author{Jeremy Fischer\\Omeed Habibelahian\\Group 17}

\begin{document}
	\maketitle
	\newpage
	
	\section{Design}
		\begin{lstlisting}
		scan through all pages
		  scan through all blocks
		    record the difference between size needed and free units available
			  If the difference is zero
			    use this spot
			    end
			 if the difference is smaller than the previous difference
			   use this location instead
		If there's no possible locations
		  create a new page
		\end{lstlisting}
		
	
	\section{Version Control Log}
		\begin{center}
			\begin{longtabu} to \textwidth {|
					X[4,l]|
					X[3,c]|
					X[8,l]|}
				\hline
				\textbf{Author} & \textbf{Date} & \textbf{Message} \\ \hline
	
				fischjer4 & 2017-11-27 & created Slob-Best-Fit folder \\ \hline
				fischjer4 & 2017-11-27 & Testing \\ \hline
				fischjer4 & 2017-11-27 & Added design for the best-fit algorithm to the writeup. Also brought in the newSlob.c which will be incorporating the best-fit algorithm \\ \hline
				fischjer4 & 2017-11-27 & Added best-fit algorithm. Created a check-page function that makes sure that the size can fit in the pages block after alignment. Edited the slob\_page\_alloc function to mirror the check-page function where it checks all blocks on all pages \\ \hline
				fischjer4 & 2017-11-28 & added missing brackets \\ \hline
				fischjer4 & 2017-11-28 & Adde linkage and syscall header files \\ \hline
				fischjer4 & 2017-11-28 & Added the system calls and the fragmentation tracking \\ \hline
				fischjer4 & 2017-11-28 & Added fragmentation test \\ \hline
				fischjer4 & 2017-11-28 & Realized to test slob, kernel space methods need to be used. So, instead there will be 5 iterations where each one makes the system calls \\ \hline
				fischjer4 & 2017-11-28 & added modified syscall modification files for quick reference \\ \hline
				fischjer4 & 2017-11-28 & update fragTest. Added syscall to allocate memory via kmalloc. added oldSlobWithFrag.c \\ \hline
				fischjer4 & 2017-11-28 & changing output of fragtest \\ \hline
				fischjer4 & 2017-11-28 & Fixed frag test missing percent output. Alos move mem\_claimed in slob.c so it ACTUALLY captures the claimed memory, not just new pages \\ \hline
				fischjer4 & 2017-11-28 & Changed unsigned long to long in syscall prototype \\ \hline
				fischjer4 & 2017-11-29 & Removed `optimization' from slob\_alloc that is no longer applicable to the best-fit algorithm \\ \hline
				fischjer4 & 2017-11-29 & Fixed `ISO forbids declaration and code' warning in slob\_alloc() \\ \hline
				fischjer4 & 2017-11-29 & Added Writeup \\ \hline
				fischjer4 & 2017-11-29 & Added Patch file \\ \hline
				
			\end{longtabu}
		\end{center}
		
	\section{Work Log}
		Our Work Log is the same as our Version Control Log above.
		One can see what was done when, as well as that we followed our design pretty tightly by looking at the commit messages.
		
	\section{Questions}
		\subsection{Main Point of the Assignment}
			The purpose of this assignment was for students to understand how to modify a component of the Linux Kernel.
			This assignment also showed students that the Linux kernel is modifiable, so one can change almost any part they want, to meet their needs.
			In this assignment, the need was to lower memory fragmentation.
			The second point was to understand how memory management in the Linux kernel works, specifically the SLOB.
			The third point was to understand how to create and use a system call.
		\subsection{Our Approach}
			First we looked at the SLOB that was currently being used, to understand how it functions.
			Once the \textit{slob\_alloc()} function was understood, we proceeded to implement the design listed in the Design section.
			Once the best-fit algorithm was implemented we needed a way to calcualate the effecieny of the modified SLOB.
			We added a\textit{ mem\_claimed} vaiable that held the summation of allocation request sizes.
			Then, in the system call \textit{sys\_amt\_mem\_free} we loop through the three lists of heap pages and record the free units in each one.
			Now that we had the the amount of memory claimed and the amount of outstanding free memory we were able to compare the two.
			The fragTest.c program calls the two system calls and calculates a percent (\textit{mem\_free / mem\_claimed}).
			This same test is applied to the original SLOB and our SLOB for comparison.
		\subsection{Ensuring Correctness}
			We ensured that our program was correct by
			\begin{enumerate}
				\item Placing \textit{printk} statements in the system call functions to ensure that the correct values were being returned
				\item The kernel booted without error
				\item The best-fit SLOB was twice as efficient as the first-fit SLOB (which is the desired outcome).
			\end{enumerate}
		
		\subsection{What We Learned}
			We learned\dots 
			\begin{itemize}
				\item How to modify the memory management sector of the Linux kernel to meet our needs 
				\item How to create and use system calls
				\item How the SLOB works
			\end{itemize}
		
		\subsection{How to Evaluate and Prove Correctness}
			\textbf{Steps to Run with Patch}
			\begin{enumerate}
				\item git clone "git://git.yoctoproject.org/linux-yocto-3.19" linux-yocto-3.19-patched
				
				\item cd linux-yocto-3.19-patched
				
				\item git checkout v3.19.2
				
				\item Apply Patch
				\begin{enumerate}
					\item cd ../
					\item patch -p0 $<$ slobBestFist.patch
				\end{enumerate}
				
				\item cp /scratch/files/core-image-lsb-sdk-qemux86.ext4 .
				\item cd linux-yocto-3.19-patched
				\item cp /scratch/files/config-3.19.2-yocto-standard .config
				\item make menuconfig
				
				\item type / 'press enter'
				\item type slob 'press enter'
				\item press 1 'press enter'
				\item check the box SLOB
				\item save and exit
				
				\item make -j4 all
				
				\item Use the \textit{screen} command to create two split-screens.
				
				\item In both screens, make sure to source the environment variable located at /scratch/files.
				\begin{enumerate}
					\item{\textbf{Screen 1:} qemu-system-i386 -gdb tcp::5517 -S -nographic -kernel arch/x86/boot/bzImage -drive file=../core-image-lsb-sdk-qemux86.ext4,if=ide -enable-kvm -usb -localtime --no-reboot --append "root=/dev/hda rw console=ttyS0 debug"}
					
					\item \textbf{Screen 2:} \$GDB
					
					\item \textbf{Screen 2:} target remote :5517
					
					\item \textbf{Screen 2:} continue
					
					\item \textbf{Screen 1:} log in as root.
					
					\item The rest of the commands below reside in \textbf{Screen 1:}
					
					\item{scp \textbf{username}@os2.engr.oregonstate.edu:\textbf{your dir}/fragTest.c /home/root}
					
					\item{gcc fragTest.c}
					\item{./a.out then view output}
					
					\item reboot (this stops qemu)
				\end{enumerate}
			\end{enumerate}

\end{document}