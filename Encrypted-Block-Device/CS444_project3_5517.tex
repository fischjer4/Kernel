\documentclass[draftclsnofoot, onecolumn, 10pt, compsoc]{IEEEtran}

\usepackage[english]{babel}
\usepackage{amsmath}
\usepackage{graphicx}
\usepackage[top=0.75in, bottom=0.75in, left=0.75in, right=0.75in]{geometry}

\usepackage{hyperref}
\usepackage{enumitem}

\usepackage{tabu}
\usepackage{longtable}

\title{\textbf{Operating Systems II}\\Fall 2017\\Homework 3}

\author{Omeed Habibelahian\\Jeremy Fischer\\Group 17}

\begin{document}
	\maketitle
	\newpage
	
	\section{Design}
		\begin{enumerate}
			\item Write a RAM disk device driver that registers it as block device
			\item Using the crypto API, add encryption to the block device such that the device driver encrypts and decrypts data when it is written and read
			\item This will be developed in the drivers/block directory
			\item This will need to be a module, in order to use a module parameter to set the key and key length.
			\item The module must then be moved in to the running VM. Kevin recommends scp.
			\item Since the Single Block Cipher encrypts one block at a time, we will loop over the number of blocks the data spans and encrypt each block sequentially
			\item The key must be set as a module parameter
			\item An incorrect key must produce gibberish
			\item Create an ext2 file system on the block device (mkfs.ext2).
		\end{enumerate}
	
	\section{Version Control Log}
		\begin{center}
			\begin{longtabu} to \textwidth {| X[4,l] | X[3,c] | X[8,l] |}
				\hline
				\textbf{Author} & \textbf{Date} & \textbf{Message} \\ \hline
				
				fischjer4 & 2017-10-31 & First Commit for Encypted Block Device \\ \hline
				
				blazerzero & 2017-10-31 & Added writeup files \\ \hline
				
				fischjer4 & 2017-11-02 & changed name. There was a typo \\ \hline
				
				fischjer4 & 2017-11-02 & Merge branch `master' of https://github.com/fischjer4/Kernel \\ \hline
				
				fischjer & 2017-11-02 & removed not needed files \\ \hline
				
				fischjer & 2017-11-02 & Merge branch `master' of https://github.com/fischjer4/Kernel \\ \hline
				
				fischjer & 2017-11-02 & adding empty crypto-block-driver.c file \\ \hline
				
				fischjer & 2017-11-02 & added cypto-block-device file \\ \hline
				
				fischjer & 2017-11-02 & renamed file \\ \hline
				
				fischjer & 2017-11-02 & added sbd skeleton to the block\_device\_crypto file \\ \hline
				
				fischjer4 & 2017-11-05 & added crypto handle init and printk statements to \_\_init function \\ \hline
				
				blazerzero & 2017-11-05 & Added \#include \textless{}linux.crypto.h\textgreater{} to block\_dev\_crypto.c \\ \hline
				
				fischjer4 & 2017-11-05 & writing encrypted data to block is completed \\ \hline
				
				fischjer4 & 2017-11-05 & Merge branch `master' of https://github.com/fischjer4/Kernel \\ \hline
				
				fischjer4 & 2017-11-05 & added decryption section \\ \hline
				
				fischjer4 & 2017-11-05 & added printk statements to print memory prior to enc/dec and after enc/dec \\ \hline
				
				fischjer4 & 2017-11-05 & added a print\_mem function which prints the elements of the buffer and memory of the block device \\ \hline
				
				blazerzero & 2017-11-05 & Included \textless{}linux/random.h\textgreater{} to initialize the key randomly. Also created new static const u8 variable for the key. \\ \hline
				
				fischjer4 & 2017-11-05 & added comments to code \\ \hline
				
				fischjer4 & 2017-11-05 & Merge branch `master' of https://github.com/fischjer4/Kernel \\ \hline
				
				fischjer4 & 2017-11-05 & set key as module\_param \\ \hline
				
				fischjer4 & 2017-11-05 & added printk statement for key so we know it was loaded properly \\ \hline
				
				fischjer4 & 2017-11-06 & added comments \\ \hline
				
				fischjer & 2017-11-06 & added Makefile, Kconfig and added them along with the module C file to a neededFiles directory \\ \hline
				
				fischjer4 & 2017-11-06 & fixed syntax error in Kconfig file \\ \hline
				
				fischjer & 2017-11-06 & fixed Kconfig line endings error \\ \hline
				
				fischjer & 2017-11-06 & changed req-\textgreater{}buffer to bio\_data(req-\textgreater{}bio) in sbd\_transfer(). req-\textgreater{}buffer is no longer a thing in this version of linux, and thus errors were produced with it \\ \hline
				
				fischjer & 2017-11-06 & fixed print\_mem to print hex, printing chars wouldn't work obviosuly \\ \hline
				
				fischjer & 2017-11-06 & fixed print --\textgreater{} printk syntax error \\ \hline
				
				fischjer & 2017-11-06 & changed print\_mem size to only 100 bytes. Printing all bytes CLUTTERED the screen \\ \hline
				
				fischjer & 2017-11-06 & fixed printk error  \\ \hline
				\\
			\end{longtabu}
		\end{center}	
	
	\section{Work Log}
		Our Work Log is the same as our Version Control Log above.
		One can see what was done when, as well as that we followed our design pretty tightly by looking at the commit messages.
		
	\section{Questions}
		\subsection{Main Point of the Assignment}
			The first main point of this assignment was to understand how a block device works, how a ramdisk drive works, and how to build and load a kernel module. The second main point of this assignment was to familiarize ourselves with the Linux Kernel's Crypto API and learn how to use this API to enable encryption and decryption of data on our block device.
		\subsection{Our Approach}
			We based our block device design off of the \href{http://blog.superpat.com/2010/05/04/a-simple-block-driver-for-linux-kernel-2-6-31/}{sbd.c} example by Pat Paterson. From there we researched how to use the Linux Crypto API and came across the "Single Block Cipher" API and thought it would be the best fit for adding encryption capabilities to our device. The Single Block Cipher only encrypts one block at a time, so we run a loop over the number of blocks the data spans and encrypt each block sequentially.
		\subsection{Ensuring Correctness}
			To ensure that the block device and the encryption/decryption code are acting as expected, we've added \textit{printk()} statements throughout the code. Once the qemu environment is booted, check the output of the \textit{printk()} statements by typing \textit{dmesg} in the command line or by viewing \textit{/var/log/messages}. The \textit{printk()} statements we've included indicate whether the device is reading or writing data, whether the block device was successfully initiated, and whether or not an initialization failure occurred anywhere in the code.
			
			\textit{dmesg} should show what the data is before and after it encrypts and decrypts.
			There are print statements that print the first 100 bytes of data that are written/read. 
			The output is in hexadecimal.
			By creating a file and writing "AA" to it, \textit{dmesg} will show that the data before encryption is 414100000 (followed by more 0's). 
			4141 in hexadecimal is AA.
			The data after encryption will be a bunch of gibberish, meaning the encryption worked correctly.
		\subsection{What We Learned}
			We learned how to\dots 
			\begin{itemize}
				\item use the Linux crypto API
				
				\item create a module
				
				\item load and unload a module
				
				\item create a block device
				
				\item mount a drive
			\end{itemize}
		\subsection{How to Evaluate and Prove Correctness}
			\textbf{Steps to Run with Patch}
			\begin{enumerate}
				\item git clone "git://git.yoctoproject.org/linux-yocto-3.19" linux-yocto-3.19-patched
				
				\item cd linux-yocto-3.19-patched
				
				\item git checkout v3.19.2
				
				\item Apply Patch
				\begin{enumerate}
					\item cp ../kernAssn3.patch drivers/block
				
					\item cd drivers/block
				
					\item patch $<$ kernAssn3.patch
				\end{enumerate}
				\item cd ../../
				
				\item cp /scratch/files/core-image-lsb-sdk-qemux86.ext4 .
				
				\item cp /scratch/files/config-3.19.2-yocto-standard .config
				
				\item make menuconfig
				
				\item Save and Exit
				
				\item make -j4 all
				
				\item cp /scratch/files/core-image-lsb-sdk-qemux86.ext4 ../
				
				\item Use the \textit{screen} command to create two split-screens.
				
				\item In both screens, make sure to source the environment variable located at /scratch/files.
				\begin{enumerate}
					\item{\textbf{Screen 1:}qemu-system-i386 -gdb tcp::5517 -S -nographic -kernel arch/x86/boot/bzImage -drive file=../core-image-lsb-sdk-qemux86.ext4,if=ide -enable-kvm -usb -localtime --no-reboot --append "root=/dev/hda rw console=ttyS0 debug"}
					
					\item \textbf{Screen 2:} \$GDB
					
					\item \textbf{Screen 2:} target remote :5517
					
					\item \textbf{Screen 2:} continue
					
					\item \textbf{Screen 1:} log in as root.
					
					\item The rest of the commands below reside in \textbf{Screen 1:}
					
					\item{scp \textbf{username}@os2.engr.oregonstate.edu:\textbf{your dir}/linux-yocto-3.19-patched/drivers/block/block\_dev\_crypto.ko /home/root}
					
					\item{Initialize the module and set the key and kylen: \textit{insmod block\_dev\_crypto.ko key="abcdefghijklmnop" keylen=16}}
					
					\item \textit{dmesg} In the output you should see that the module was initialized and the key was set
					
					\item \textit{dmesg -c} Clear the kernel 's output clutter
					
					\item Partition the disk:\textit{ fdisk /dev/crptblkd0}
						Follow these instructions when prompted
						\begin{itemize}
							\item Command (m for help): \textbf{n}
					
							\item Command action  e extended  p primary partition (1-4): \textbf{p}
					
							\item Partition number (1-4): \textbf{1}
					
							\item First sector(1-1023, default 1): \textbf{1}
					
							\item Last sector, +sectors or +size{K,M,G,T,P} (1-1023, default 1023): \textbf{$<$press enter$>$}
					
							\item Command (m for help): \textbf{w}
						\end{itemize}
					
					\item Create filesystem: \textit{mkfs.ext2 /dev/crptblkd0p1}
					
					\item Mount filesystem: \textit{mount /dev/crptblkd0p1 /mnt}
				\end{enumerate}
				
				\item Now that the module is loaded and the drive is mounted, we will assert its correctness
				\begin{itemize}
					\item \textit{echo AA $>$ /mnt/file1}
				
					\item \textit{ls -l /mnt} You should see file1 now
				
					\item 
						\textit{dmesg} It may take a few seconds, but you will see that "Before ENcryption" the data reads 41410000...0000....000. 
						The output is in hex, and 4141 is AA.
						in the "After ENcryption" section you can see that the data was encrypted to giberish now
					
					\item \textit{cat /mnt/file1} You will see AA as the output
				\end{itemize}
				
				\item Unmount the drive: \textit{umount /mnt}
				
				\item Uninstall the module: \textit{rmmod block\_dev\_crypto}
				
				\item reboot (this stops qemu)
			\end{enumerate}

\end{document}